\documentclass[../fem.tex]{subfile}

\begin{document}
\section{Problem}%
\label{sec:problem}

\subsection{Setup}%
\label{sub:setup}

For this problem we consider the matrices that define the $L$ operator to be
\begin{align*}
  A=\begin{bmatrix}
    1 & 0\\ 0 & 1
  \end{bmatrix}\qquad
  B=\begin{bmatrix}
    0 \\ 0
  \end{bmatrix}\qquad
  C=0.
\end{align*}
Thus we can write the operator $L$ to be
\begin{align*}
  L=\ppder{}{x}+\ppder{}{y}.
\end{align*}

\subsection{Problem Statement}%
\label{sub:problem_statement}

We will consider a problem that is time independent, and so will have no time
derivative. This means that the formulation of the problem statement will be
\begin{align*}
  Lu=f.
\end{align*}

The problem that we consider is
\begin{align}
  \ppder{u}{x}+\ppder{u}{y}=-8\sin(2x)\sin(2y).
\end{align}
With the boundary conditions defined by
\begin{align*}
  \partial u(x, y, t)=\sin(2x)\sin(2y).
\end{align*}

\subsection{Analytical}%
\label{sub:analytical}

This problem was specially constructed such that the solution is
\begin{align*}
  u=\sin(2x)\sin(2y),
\end{align*}
we will verify that this is indeed a solution to the problem.

\begin{align*}
  \ppder{}{x}\left[\sin(2x)\sin(2y)\right]&+\ppder{}{y}\left[\sin(2x)\sin(2y)\right]\\
                                          &=\pder{}{x}\left[-2\cos(2x)\sin(2y)\right]+\pder{}{y}\left[-2\sin(2x)\cos(2y)\right]\\
                                          &=-4\sin(2x)\sin(2y)-4\sin(2x)\sin(2y)\\
                                          &=-8\sin(2x)\sin(2y).
\end{align*}
Thus we have proven that a solution to this problem is $u=\sin(2x)\sin(2y)$.

\subsection{Implementation}%
\label{sub:implementation}

We have constructed the finite element analysis, to accept a \texttt{Lua}
script file. The Lua script defines all of the necessary information for the
finite element analysis to proceed. For this problem we define the problem Lua
script below.

\code{lua}{snippets/r1.lua}

\subsection{Unit Square}%
\label{sub:unit_square}

First we provide the findings on the unit square. We will compare the results
of the unit square on four different mesh resolutions. These resolutions are
$0.05$, $0.01$, $0.005$, $0.001$. Recall that the mesh resolution is the maximum
area of the individual triangles of the mesh. We will denote the mesh
resolution using $\mu$. The results of the computations are provided below.

\begin{figure}[htpb]
  \centering
  \begin{subfigure}{0.4\textwidth}
    \centering
    \includegraphics[width=0.8\linewidth]{figures/r1a/tri.png}
    \caption{$\mu=0.05$, $n=77$}
  \end{subfigure}
  \begin{subfigure}{0.4\textwidth}
    \centering
    \includegraphics[width=0.8\linewidth]{figures/r1b/tri.png}
    \caption{$\mu=0.01$, $n=338$}
  \end{subfigure}
  \begin{subfigure}{0.4\textwidth}
    \centering
    \includegraphics[width=0.8\linewidth]{figures/r1c/tri.png}
    \caption{$\mu=0.005$, $n=666$}
  \end{subfigure}
  \begin{subfigure}{0.4\textwidth}
    \centering
    \includegraphics[width=0.8\linewidth]{figures/r1d/tri.png}
    \caption{$\mu=0.001$, $n=6249$}
  \end{subfigure}
  \caption{Triangular meshes of $\Box^2$ used for solving the problem.}
  \label{fig:b2_mesh}
\end{figure}

\begin{figure}[htpb]
  \centering
  \begin{subfigure}{0.5\textwidth}
    \centering
    \includegraphics[width=0.8\linewidth]{figures/r1a/func.png}
    \caption{Analytical Solution}
  \end{subfigure}

  \begin{subfigure}{0.4\textwidth}
    \centering
    \includegraphics[width=0.8\linewidth]{figures/r1a/approx.png}
    \caption{$\mu=0.05$}
  \end{subfigure}
  \begin{subfigure}{0.4\textwidth}
    \centering
    \includegraphics[width=0.8\linewidth]{figures/r1b/approx.png}
    \caption{$\mu=0.01$}
  \end{subfigure}
  \begin{subfigure}{0.4\textwidth}
    \centering
    \includegraphics[width=0.8\linewidth]{figures/r1c/approx.png}
    \caption{$\mu=0.005$}
  \end{subfigure}
  \begin{subfigure}{0.4\textwidth}
    \centering
    \includegraphics[width=0.8\linewidth]{figures/r1d/approx.png}
    \caption{$\mu=0.001$}
  \end{subfigure}
  \caption{Side by side comparison of the analytical and numerical solutions.}
  \label{fig:r1_soln}
\end{figure}

It can clearly be seen in figure \ref{fig:r1_soln}, that there is little to no
difference, between the analytical and the numerical solutions. The
quantitative comparisons of the different methods is provided in table
\ref{tab:r1}.

\begin{table}[htpb]
  \centering
  \caption{Times and error required for each of the mesh refinement levels.}
  \label{tab:r1}
  \begin{tabular}{c c}
    \begin{tabular}{r l}
      \multicolumn{2}{c}{$\mu=0.05$}\\\hline\hline
      Stage & Time (HH:MM:SS)\\\hline
      Mesh & 00:00:00.002735\\
      Matrices & 00:00:00.558476\\
      Forcing & 00:00:00.008525\\
      Solving & 00:00:00.059320\\\hline
      Total & 00:00:00.629056\\
      Error & 0.111038\\\hline
    \end{tabular}
    &
    \begin{tabular}{r l}
      \multicolumn{2}{c}{$\mu=0.01$}\\\hline\hline
      Stage & Time (HH:MM:SS)\\\hline
      Mesh & 00:00:00.007338\\
      Matrices & 00:00:02.785410\\
      Forcing & 00:00:00.047460\\
      Solving & 00:00:05.320913\\\hline
      Total & 00:00:08.161121\\
      Error & 0.027810\\\hline
    \end{tabular}\\[2cm]
    \begin{tabular}{r l}
      \multicolumn{2}{c}{$\mu=0.005$}\\\hline\hline
      Stage & Time (HH:MM:SS)\\\hline
      Mesh & 00:00:00.026886\\
      Matrices & 00:00:05.753090\\
      Forcing & 00:00:00.084677\\
      Solving & 00:00:44.845096\\\hline
      Total & 00:00:50.709749\\
      Error & 0.010363\\\hline
    \end{tabular}
    &
    \begin{tabular}{r l}
      \multicolumn{2}{c}{$\mu=0.001$}\\\hline\hline
      Stage & Time (HH:MM:SS)\\\hline
      Mesh & 00:00:00.048287\\
      Matrices & 00:00:31.228624\\
      Forcing & 00:00:00.454135\\
      Solving & 01:21:50.405917\\\hline
      Total & 01:22:22.136963\\
      Error & 0.002487\\\hline
    \end{tabular}
  \end{tabular}
\end{table}

\subsection{Unit Disk}%
\label{sub:unit_disk}

Now we provide the findings on the unite disk. We will compare the results on
the unit disk using the same changes that we made for the unit square. The
results of the computations are provided below.

\begin{figure}[htpb]
  \centering
  \begin{subfigure}{0.4\textwidth}
    \centering
    \includegraphics[width=0.8\linewidth]{figures/c1a/tri.png}
    \caption{$\mu=0.05$, $n=93$}
  \end{subfigure}
  \begin{subfigure}{0.4\textwidth}
    \centering
    \includegraphics[width=0.8\linewidth]{figures/c1b/tri.png}
    \caption{$\mu=0.01$, $n=273$}
  \end{subfigure}
  \begin{subfigure}{0.4\textwidth}
    \centering
    \includegraphics[width=0.8\linewidth]{figures/c1c/tri.png}
    \caption{$\mu=0.005$, $n=525$}
  \end{subfigure}
  \begin{subfigure}{0.4\textwidth}
    \centering
    \includegraphics[width=0.8\linewidth]{figures/c1d/tri.png}
    \caption{$\mu=0.001$, $n=2501$}
  \end{subfigure}
  \caption{Triangular meshes of $D^2$ used for solving the problem.}
  \label{fig:d2_mesh}
\end{figure}

\begin{figure}[htpb]
  \centering
  \begin{subfigure}{0.5\textwidth}
    \centering
    \includegraphics[width=0.8\linewidth]{figures/c1a/func.png}
    \caption{Analytical Solution}
  \end{subfigure}

  \begin{subfigure}{0.4\textwidth}
    \centering
    \includegraphics[width=0.8\linewidth]{figures/c1a/approx.png}
    \caption{$\mu=0.05$}
  \end{subfigure}
  \begin{subfigure}{0.4\textwidth}
    \centering
    \includegraphics[width=0.8\linewidth]{figures/c1b/approx.png}
    \caption{$\mu=0.01$}
  \end{subfigure}
  \begin{subfigure}{0.4\textwidth}
    \centering
    \includegraphics[width=0.8\linewidth]{figures/c1c/approx.png}
    \caption{$\mu=0.005$}
  \end{subfigure}
  \begin{subfigure}{0.4\textwidth}
    \centering
    \includegraphics[width=0.8\linewidth]{figures/c1d/approx.png}
    \caption{$\mu=0.001$}
  \end{subfigure}
  \caption{Side by side comparison of the analytical and numerical solutions.}
  \label{fig:d1_soln}
\end{figure}

It can clearly be seen in figure \ref{fig:d1_soln}, that there is little to no
difference, between the analytical and the numerical solutions. This is
especially true for the higher refinement of the mesh. The quantitative
comparisons of the different methods is provided in table
\ref{tab:c1}.

\begin{table}[htpb]
  \centering
  \caption{Times and error required for each of the mesh refinement levels.}
  \label{tab:c1}
  \begin{tabular}{c c}
    \begin{tabular}{r l}
      \multicolumn{2}{c}{$\mu=0.05$}\\\hline\hline
      Stage & Time (HH:MM:SS)\\\hline
      Mesh & 00:00:00.002921\\
      Matrices & 00:00:01.177174\\
      Forcing & 00:00:00.019072\\
      Solving & 00:00:00.139923\\\hline
      Total & 00:00:01.33909\\
      Error & 0.016088\\\hline
    \end{tabular}
    &
    \begin{tabular}{r l}
      \multicolumn{2}{c}{$\mu=0.01$}\\\hline\hline
      Stage & Time (HH:MM:SS)\\\hline
      Mesh & 00:00:00.004376\\
      Matrices & 00:00:03.800784\\
      Forcing & 00:00:00.058017\\
      Solving & 00:00:4.476645\\\hline
      Total & 00:00:08.379206\\
      Error & 0.003888\\\hline
    \end{tabular}\\[2cm]
    \begin{tabular}{r l}
      \multicolumn{2}{c}{$\mu=0.005$}\\\hline\hline
      Stage & Time (HH:MM:SS)\\\hline
      Mesh & 00:00:00.006625\\
      Matrices & 00:00:07.717563\\
      Forcing & 00:00:00.116296\\
      Solving & 00:00:32.456541\\\hline
      Total & 00:00:40.29665\\
      Error & 0.002074\\\hline
    \end{tabular}
    &
    \begin{tabular}{r l}
      \multicolumn{2}{c}{$\mu=0.001$}\\\hline\hline
      Stage & Time (HH:MM:SS)\\\hline
      Mesh & 00:00:00.025953\\
      Matrices & 00:00:38.557897\\
      Forcing & 00:00:00.587212\\
      Solving & 01:02:43.771275\\\hline
      Total & 01:03:22.942397\\
      Error & 0.000509\\\hline
    \end{tabular}
  \end{tabular}
\end{table}

Clearly for the simple cases like this one, the method of finite element
analysis works extremely well, and quickly for smaller numbers of triangles.

For both of these case increasing the mesh refinement, greatly increased the
number of vertices. The number of vertices is on the order
of$\sim\frac{3}{\mu}$. This means that as $\mu$ is decreases the number of
vertices explodes. This will indicate that the time required for computation
will also explode.

We observe that the time of computation does indeed directly correlate the
number of vertices in the mesh, and is on the order of $\sim N^3$. This means
that a small decrease in mesh size, means that the time will grow greatly,
making the approximation take significantly longer to compute.

Finally we notice that the error decreases almost linearly. This will greatly
depend on the problem that is being computed. For this problem, the
approximation was already very accurate, and so the increase in mesh
refinement, did not provide much of a benefit in the error from the analytical
solution.

With this analysis of the implementation, it is clear that larger meshes should
be avoided as much as possible, as they result in drastically longer computation
times. For problems with few changes are very well handled by this method, and
do not require a very fine mesh. However, problems with many changes will be
handled poorly, and will require extremely small mesh triangles. This can be
considered in reference to a 2D example.

Consider $\sin(x)$ and $\sin(10x)$, because of the higher frequency of
$\sin(10x)$ it would be less well approximated, without many many triangles,
but $\sin(x)$ would be very easily approximated with only a few triangles. So
with a ``higher frequency'' function a finer mesh would be required to achieve
acceptable error.

\end{document}
