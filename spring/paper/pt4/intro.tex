Thats\documentclass[../fem.tex]{subfile}

\begin{document}
\section{Introduction}%
\label{sec:introduction_4}

In this part, we present some examples of the solutions generated by the
implementation and compare them to the analytical solutions of the problem. We
do this to validate the accuracy of the approximations generated using the
method of finite elements.

We present two problems in this section and use our implementation of Finite
Element Analysis to solve for the solution. We present the findings on two
different domains, the Unit Disk $D^2$ and the Unit Square $\Box^2$.

We will also comment on the time of computation for the numeric approximation,
and how that changes. The time of computation is broken into several stages.
\begin{description}
  \item[Mesh Generation] This is the time it takes to construct the mesh from
    the PSLG, and refine it to the user specifications.
  \item[Construct Matricies] This is the time required to construct the global
    $G$ and $M$ matrices.
  \item[Construct Forcing] This is the time required to construct the global
    $F$ matrix for the current time step.
  \item[Solving] This is the time required to solve the system of linear
    equations, which in the time-independent case is $MU=F$.
\end{description}

The grain size of the mesh is also very important, in the computation of the
numerical approximation. So we will test several mesh resolutions to compare
how a finer mesh will improve the approximation.

In order to compare the analytical solution to the numerical solution, we
utilize the difference between the two functions by using the maximum
difference at any given point. This expression can be found by
\begin{align*}
  E\equiv\sup\left(u_\text{approx}(x,y)-u_\text{soln}(x,y)\right)\quad\forall\quad
    (x,y)\in\Omega.
\end{align*}

\end{document}
