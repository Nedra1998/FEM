\documentclass[../fem.tex]{subfile}

\begin{document}
\section{Conditions}%
\label{sec:conditions}

For our proposed problem statement, we will require some conditions in order to
make it possible to determine a solution to the problem. The conditions that we
will require are boundary conditions and if we are considering a time dependent
system, we will also need an initial condition.

\subsection{Boundary Conditions}%
\label{sub:boundary_conditions}

We must prescribe boundary conditions on our problem statement, to ensure that
there is a single solution. For now we will only proceed utilizing Dirichlet
boundary conditions. It should be noted that other conditions are possible with
this method, such as Neumann, and Newton boundary conditions. These other
boundary conditions require more changes to the formulation of the problem
later on and require extra modifications to the implementation. Because of the
extra alterations that they would necessitate, we will only focus on the
Dirichlet boundary conditions.

For the Dirichlet boundary conditions we must define the value of our solution
at every point on the boundary. We define this as
\begin{align*}
  \partial u(\vec{x},t)\quad\vec{x}\in\partial\Omega.
\end{align*}
We will use this notation to denote the function defining the values of the
solution to the problem along the boundary at time $t$. This can be written as
\begin{align*}
  u(\partial\Omega,t)\equiv\partial u(\partial\Omega,t).
\end{align*}
It is important to note that there is no restrictions to the form of $\partia
u$. The user defined function can take any form, and thus requiring the
solution to also fit any form of boundary conditions.

\subsection{Initial Conditions}%
\label{sub:initial_conditions}

If the problem being considered were to be time dependent, then it is also
necessary to define an initial condition to the problem. That is to say that it
is required to provide some function that defines the value of $u$ at $t=0$. We
define this function as
\begin{align*}
  u_0(\vec{x})\quad\vec{x}\in\Omega.
\end{align*}
We use this notation to denote the function defining the values of the
solution to the problem, that is to say
\begin{align*}
  u(\Omega,t=0)\equiv u_0(\Omega).
\end{align*}

Note that this function also has no restrictions other that it must be defined
on the entirety of the domain $\Omega$, although it need not be continuous on
the domain.

If the problem being considered is not time dependent, than the initial
condition can be ignored, and only the boundary conditions must be specified.

Using these prescribed conditions, both the boundary and the initial
conditions, it is now possible to utilize finite element methods to attain an
approximation of the solution to the problem statement.

\end{document}
