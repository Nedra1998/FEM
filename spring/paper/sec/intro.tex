\documentclass[../fem.tex]{subfile}

\begin{document}
\section{Introduction}%
\label{sec:introduction}

Finite element methods is a computational process used for the determination of
an approximate solution to a given partial differential equation, with given
forcing term and boundary conditions. Finite element analysis is the practical
implementation of finite element methods computationally on a physical problem
in order to ascertain more insight into the physical system in question. We
will first explain finite element methods, and provide a thorough introduction
into the mathematics behind the methods. Then we will also provide an
explanation of finite element analysis, demonstrating the mathematics of FEM
with respect to an example problem, and develop a computational program that we
will use to solve the partial differential equation using FEM.

Finite element methods is very useful, due to the fact that there are only a
few situations where a solution to a partial differential equation can be found
analytically, so we need methods to find a close approximation of the actual
solution. Currently there are a handful of methods that can be used for this
purpose, however finite element methods are currently the best methods that we
know of. Meaning that the error between the analytical solution and the
solution constructed by finite element analysis converges to zero faster than
other methods.

The process that we utilize for constructing the finite element methods is
called the Galerkin method, and is presented by \cite{KH}. We will closely
follow the method outlined by their paper. For this introduction we will limit
the scope to two dimensions, but the mathematics can be easily generalized to
higher dimensions, and we will make note of the sections that would have to
change for higher dimension problems.

We divide this paper into two major parts. Part 1 explains the process of FEM,
and demonstrates the mathematics and the concepts behind the method, and how it
can be used to solve the partial differential equations. Part 2 explains the
process of FEA, and discusses the algorithms, data structures, and
optimizations that can be done to construct a program to efficiently compute
the solution utilizing the previously discussed FEM.

\end{document}
