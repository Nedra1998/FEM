\documentclass[../fem.tex]{subfile}

\begin{document}
\section{Dirichlet Boundary Conditions}%
\label{sec:dirichlet_boundary_conditions}

Now that the global system of equations has been constructed, we are able to
impose the predefined boundary conditions to the system of equations. As
discussed previously, we will only consider the Dirichlet boundary conditions
in this explanation.

The global system of equations can be written as
\begin{align*}
  G\partial_t U=MU+F
\end{align*}
for the time dependent problem, and as
\begin{align*}
  MU&=F
\end{align*}

In the current state this system of equations is not solvable, because in the
construction of the matricies, we included the basis elements that reside on
the boundary. As discuessed previously, these basis elements are not a part of
the approximation, but must be used inorder to apply the boundary conditions.
At this point, we assign the boundary conditions to the corresponding equations
for these boundary basis elements.

To do this we enforce that the associated value of the $F$ vector for each
$\ph_i\in\partial \Omega$, will be the value of the boundary at that point. As
we defined in section \ref{sub:boundary_conditions}, this means that
\begin{align*}
  F_i=\partial u\left(\left<X_i,Y_i\right>,t\right),
\end{align*}
where $X_i$ and $Y_i$ are the $x$ and $y$ coordinates of the $i$th vertex of
the mesh, we will denote this as $\vec{x}_i$. In order to apply this boundary
coniditon to the $M$ matrix, we must also restrict that
\begin{align*}
  M_{ij}=\begin{cases}
    0 & i\neq j\\
    1 & i=j
  \end{cases}\quad\forall j=1,\ldots,N.
\end{align*}
This expression can be rewritten using the Kroneker delta function, thus the
expression for $M_{ij}$ becomes
\begin{align*}
  M_{ij}=\delta_{ij}\quad j=1,\ldots,N.
\end{align*}

This means that taking the expression for $M$ and $F$ in a general form to be
\begin{align*}
   \begin{pmatrix}
     M_{11} & M_{12} & \cdots & M_{1N} \\
     M_{21} & M_{22} & \cdots & M_{2N}\\
     \vdots & \vdots & \ddots & \vdots \\
     M_{N1} & M_{N2} & \cdots & M_{nn}
   \end{pmatrix}\quad
   \begin{pmatrix}
      F_1 \\
      F_2 \\
      \vdots \\
      F_N
   \end{pmatrix},
\end{align*}
we can demonstrate the process of applying the boundary conditions. However,
applying the boundary conditions will be dependent on the mesh that is being
considered. For this example we will consider some arbitrary $i$'s to act as
boundary verticies. Applying the process for the enforcement of the boundary
conditions, we find that the expression of $M$ and $F$ become
\begin{align*}
   \begin{pmatrix}
     1 & 0 & \cdots & 0 & 0\\
     M_{21} & M_{22} & \cdots & M_{2(N-1)} & M_{2N}\\
     \vdots & \vdots & \ddots & \vdots & \vdots \\
     0 & 0 & \cdots & 1 & 0\\
     M_{N1} & M_{N2} & \cdots & M_{N(N-1)} & M_{NN}
   \end{pmatrix}\quad
   \begin{pmatrix}
      \partial u(\vec{x}_1,t)\\
      F_2 \\
      \vdots \\
      \partial u(\vec{x}_{(N-1)},t)\\
      F_N
   \end{pmatrix}.
\end{align*}

In this example the $1$ and the $N-1$ verticies of the mesh lie on the
boundary, and so the equations associated with those indicies have been
rewritten.


\end{document}
