\documentclass[../fem.tex]{subfile}

\begin{document}

\section{Variational Formulation}%
\label{sec:variational_formulation}

In order to construction the variational formulation of our problem statement,
we must first introduce the concept of a residual. The residual is the
construction of a function $R$ such that when our approximation is the exact
solution then $R=0$. Using the residual we now want to construct a function
that minimizes the value of the residual. We will construct the residual of
equation \ref{eq:prob} as
\begin{align*}
  R(\vec{x},t)=\pder{u}{t}-Lu-f
\end{align*}
Clearly by the construction of $R$, $R(\vec{x},t)=0$ if and only if $u$ is the
exact solution to the partial differential equation. Now we multiply the
residual by some arbitrary test function, and integrate over our domain
\begin{align*}
  \int_\Omega w(\vec{x})R(\vec{x},t)d\vec{x}.
\end{align*}

For the exact solution $u$ this will always be zero, because $R(\vec{x},t)=$
for all $\vec{x}\in\Omega$. Because of this we are able to select any test
function $w$, and the weighted residual will still be zero, given the exact
solution. However, we will restrict our test functions to ones that are zero
along the boundaries $\partial\Omega$, this will significantly simply our
computations in the future.

We now expand the weighted residual to obtain
\begin{align*}
  \int_\Omega w\pder{u}{t}d\vec{x}=\int_\Omega wLud\vec{x}+\int_\Omega
  wfd\vec{x}.
\end{align*}
Expanding our the $L$ operator on $u$ and splitting the integral over the sum,
we find
\begin{align*}
  \int_\Omega w\pder{u}{t}d\vec{x}=
  \sum_{\alpha,\beta=1}^2\int_\Omega
  wA^{\alpha\beta}\pder{}{x^\alpha}\left[\pder{u}{x^\beta}\right]d\vec{x}
  +\sum_{\gamma=1}^2\int_\Omega wB^\gamma\pder{u}{x^\gamma}d\vec{x}
  +\int_\Omega wCud\vec{x}
  +\int_\Omega wfd\vec{x}.
\end{align*}
We notice that $w$ is always independent of time so that the first integral can
be rewritten as
\begin{align*}
  \int_\Omega w\pder{u}{t}d\vec{x}&=\pder{}{t}\int_\Omega wud\vec{x}.
\end{align*}
We can now apply integration by parts on the second integral, and rewrite this
as
\begin{align*}
  \int_\Omega
  wA^{\alpha\beta}\pder{}{x^\alpha}\left[\pder{u}{x^\beta}\right]d\vec{x}
  =\int_{\partial\Omega}A^{\alpha\beta}w\pder{u}{x^\beta}d\vec{x}
  -\int_\Omega \pder{A^{\alpha\beta}w}{x^\alpha}\pder{u}{x^\beta}d\vec{x}.
\end{align*}
Since our restriction on the selection of of test functions $w$, we know that
$w(\vec{x})=0\ \forall \vec{x}\in\partial\Omega$. Thus the integral over the
boundary must be zero. Using this simiplification, we obtain the representation
of the continuous variational formulation of the problem to be
\begin{align}\label{eq:variational}
  \pder{}{t}\int_\Omega wud\vec{x}=
  -\sum_{\alpha,\beta=1}^2\int_\Omega
  \pder{A^{\alpha\beta}w}{x^\alpha}\pder{u}{x^\beta}d\vec{x}
  +\sum_{\gamma=1}^2\int_\Omega B^\gamma w\pder{u}{x_\gamma}d\vec{x}
  +\int_\Omega Cwud\vec{x}
  +\int_\Omega wfd\vec{x}.
\end{align}


\end{document}
