\documentclass[../fem.tex]{subfile}

\begin{document}
\section{Discretization}%
\label{sec:discretization}

Now that we have a variational formulation of the partial differential
equation, we wish to discretize the equation, so that it is possible to apply
the method of finite elements.

In order to discretize the problem, we must have an understanding of the space
that the solution resides in. It is clear that the solution to the equation
will be in the Sobolev space $W^{1,2}(\Omega)$. The Sobolev space is the set of
functions functions that are infinitely differentiable. In this case we can tell
that our solution resides in the specific Sobolev space $W^{1,2}(\Omega)$. It
is also important to mention that this can be written as $H^1(\Omega)$, because
this Sobolev space is also a Hilbert space. A Hilbert space is one that possess
the structure for inner product. Thus we are able to conclude that our solution
exists in a space that supports inner product, and is infinitely
differentiable.

Since the process of finite element methods is inherently intended for
computation, problems arise when infinities come into play. Since the span of
the Sobolev space $H^1(\Omega)$ is infinite, we are required to consider a
subspace of it. This will be the subspace of the Sobolev space that our
approximation of the solution lives in, we will call this subspace
$\wt{V}\subset H^1(\Omega)$.

We will construct a finite set of basis functions for the subspace $\wt{V}$ to
be $\left\{\ph_j\right\}$. For now we will leave these basis functions as
arbitrary functions, but the process for the construction of the basis
functions will be presented in a later section.

The approximation of $u$ we define as $u$, since the approximation should
converge to the exact solution, utilizing the same notation does not cause any
issues. We construct this approximation such that it is an element of the
subspace of the Sobolev space as we have defined. That is to say
\begin{align*}
  u\in\wt{V}\subset H^1(\Omega)
\end{align*}
Since our approximation is
constructed in $\wt{V}$, then it must be expressible as a linear combination of
the basis function $\ph_j$ of the finite subspace $\wt{V}$ that $\wt{u}$ is an
element of. Thus we can express $\wt{u}$ as
\begin{align}\label{eq:approx}
  u=\sum_{j=1}^N u_j(t)\ph_j(\vec{x}),
\end{align}
where $u_j(t)$ is a function of time that once it is found, it is possible to
construct the final approximation. Using this expression for our approximation
the goal of the method is to determine the functions $u_j(t)$.

We can now substitute our expression for $u$ into the continuous variational
formulation of the problem (\ref{eq:variational}).
\begin{align}\label{eq:var_exp_b}
  \pder{}{t}\int_\Omega w\left[\sum_{j=1}^Nu_j\ph_j\right]d\vec{x}&=
  -\sum_{\alpha,\beta=1}^2\int_\Omega\pder{A^{\alpha\beta}w}{x^\alpha}
  \pder{}{x^\beta}\left[\sum_{j=1}^Nu_j\ph_j\right]d\vec{x}\\
  &+\sum_{\gamma=1}^2\int_\Omega B^\gamma w
  \pder{}{x^\gamma}\left[\sum_{j=1}^Nu_j\ph_j\right]d\vec{x}
  +\int_\Omega Cw\left[\sum_{j=1}^Nu_j\pi_j\right]d\vec{x}
  +\int_\Omega wfd\vec{x}.\nonumber
\end{align}

It is possible to rewrite this expression, with several simplifications. First
is to note that since $w$ and $\ph_j$ are independent of $t$, and $u$ is
independent of $\vec{x}$. Using this it is possible to rewrite the first term,
by moving constants with respect to the integral ($u_j$) out of the integral,
and then to move constants with respect to the derivative (the integral) out of
the derivative. Thus we can write the first term as
\begin{align*}
  \pder{}{t}\int_\Omega w\left[\sum_{j=1}^Nu_j\ph_j\right]dx\vec{x}
  =\sum_{j=1}^N\left[\pder{u_j}{t}\int_\Omega w\ph_jd\vec{x}\right].
\end{align*}
Similar process can be applied to the other terms in equation
\ref{eq:var_exp_b}. By pulling the constants out of the integrals, and
splitting the integral over the summation, we are able to rewrite the expression
as
\begin{align}\label{eq:var_exp_c1}
  \sum_{j=1}^N\left[\pder{u_j}{t}\int_\Omega w\ph_jd\vec{x}\right]&=
  -\sum_{j=1}^N\left[u_j\sum_{\alpha,\beta=1}^2
    \int_\Omega\pder{A^{\alpha\beta}w}{x^\alpha}
    \pder{\ph_j}{x^\beta}d\vec{x}\right]\\
                                                                  &+\sum_{j=1}^N\left[u_j\sum_{\gamma=1}^2\int_\Omega B^\gamma w
    \pder{\ph_j}{x^\gamma}d\vec{x}\right]
  +\sum_{j=1}^N\left[u_j\int_\Omega
    Cw\ph_jd\vec{x}\right]
  +\int_\Omega wfd\vec{x}.\nonumber
\end{align}

Now, since we have not placed any restrictions on $w$, and it can be any
function, we will chose the basis functions $\ph_i$ as our test functions.
Substituting that into the expression we obtain the discretized version of the
variational formulation of the problem.
\begin{align}\label{eq:var_exp_c}
  \sum_{j=1}^N\left[\pder{u_j}{t}\int_\Omega \ph_i\ph_jd\vec{x}\right]&=
  -\sum_{j=1}^N\left[u_j\sum_{\alpha,\beta=1}^2
    \int_\Omega\pder{A^{\alpha\beta}\ph_i}{x^\alpha}
    \pder{\ph_j}{x^\beta}d\vec{x}\right]\\
                                                                      &+\sum_{j=1}^N\left[u_j\sum_{\gamma=1}^2\int_\Omega B^\gamma \ph_i
    \pder{\ph_j}{x^\gamma}d\vec{x}\right]
  +\sum_{j=1}^N\left[u_j\int_\Omega
    C\ph_i\ph_jd\vec{x}\right]
  +\int_\Omega \ph_ifd\vec{x}.\nonumber
\end{align}


It can be useful to examine this equation using the inner products as they act
upon functions. That is to say, that we use the $L^2$ inner product defined as
\begin{align*}
  \braket{f}{g}_{L^2}=\int_\Omega fgd\vec{x}.
\end{align*}
Using this notation, the expression can be rewritten to be
\begin{align*}
  \sum_{j=1}^N\braket{\ph_i}{\ph_j}\pder{u_j}{t}&=
  -\sum_{j=1}^N\sum_{\alpha,\beta=1}^2\braket{\pder{}{x^\alpha}\left[A^{\alpha\beta}\ph_i\right]}{\pder{\ph_j}{x^\beta}}u_j\\
  &+\sum_{j=1}^N\sum_{\gamma=1}^2\braket{B^\gamma\ph_i}{\pder{\ph_j}{x^\gamma}}u_j
  +\sum_{j=1}^N\braket{C\ph_i}{\ph_j}u_j
  +\braket{\ph_i}{f}.
\end{align*}
It is now possible to factor out the $u_j$ from the right hand side summations.
This will provide the equation
\begin{align*}
  \sum_{j=1}^N\braket{\ph_i}{\ph_j}\pder{u_j}{t}&=
  \left[-\sum_{j=1}^N\sum_{\alpha,\beta=1}^2\braket{\pder{}{x^\alpha}\left[A^{\alpha\beta}\ph_i\right]}{\pder{\ph_j}{x^\beta}}\right.\\
                                                &+\sum_{j=1}^N\sum_{\gamma=1}^2\braket{B^\gamma\ph_i}{\pder{\ph_j}{x^\gamma}}
  \left.+\sum_{j=1}^N\braket{C\ph_i}{\ph_j}\right] u_j
  +\braket{\ph_i}{f}.
\end{align*}

Since it is possible to replace $w$ with \textit{any} of the basis functions,
that means that $i=1,2,\ldots,N$. This provides us with a system of equations,
which is then expressible in terms of matrices. The matrix representation of
the system is
\begin{align}\label{matrix}
  G\partial_t U=\left(\wb{A}+\wb{B}+\wb{C}\right)U+F.
\end{align}
Where $U$ is a vector in $\R^N$ composed of the $u_j$ functions. We use the
notation $\partial_t$ to denote the operator of taking the partial derivative
of the $U$ vector with respect to time, element wise. The elements of $G$,
$\wb{A}$, $\wb{B}$, $\wb{C}$, and $F$ are defined as
\begin{align*}
  G_{ij}&=\braket{\ph_i}{\ph_j}\quad &i,j&=1,\ldots,N\\
  \wb{A}^{ij}&=-\sum_{\alpha,\beta=1}^2\braket{\pder{}{x^\alpha}\left[A^{\alpha\beta}\ph_i\right]}{\pder{\ph_j}{x^\beta}}
             &i,j&=1,\ldots,N\\
  \wb{B}^{ij}&=\sum_{\gamma=1}^2\braket{B^\gamma\ph_i}{\pder{\ph_j}{x^\gamma}}
             &i,j&=1,\ldots,N\\
  \wb{C}^{ij}&=\braket{C\ph_i}{\ph_j} &i,j&=1,\ldots,N\\
  F&=\braket{\ph_i}{f} &i,j&=1,\ldots,N.
\end{align*}

It is often simpler to consider a matrix $M$ such that
$M=\wb{A}+\wb{B}+\wb{C}$, then our expression for the matrix form of the system
of equations becomes
\begin{align}\label{eq:matrix_simp}
   G\partial_tU=MU+F.
\end{align}

This matrix representation is how we will continue to consider the problem. For
most cases this is the system of equations that will be found. However, this
expression is only sufficient for Dirichlet boundary conditions, as the
integral over the boundary for other boundary conditions would not be zero.
Thus there would be additional terms in the system of equations. This is not an
issue with the Dirichlet boundary conditions, so we do not need to consider any
additional terms in the matrix representation.

If the problem that is being considered were to be time independent, then the
matrix formulation of the system of equations would be
\begin{align*}
   MU=F.
\end{align*}
We will continue to consider both the time dependent and the time independent
problems, as both are useful in different situations. However, they are solved
in different ways, some simplifications can be made with one but not the other.

\end{document}
