\documentclass[../fem.tex]{subfile}

\begin{document}
\section{Problem Statement}%
\label{sec:problem_statement}

Throughout this paper and our sample code, we implement finite element methods
using the general form of the heat equation as our differential equation to
consider. We define the domain to be $\Omega$ such that $\Omega \in \R^2$ is a
bounded 2D domain with boundary $\Gamma$.

The general form of the heat equation takes the form
\begin{align*}
  \pder{u}{t}=Lu+f,
\end{align*}
where $L$ is an operator of the form
\begin{align*}
  L=\sum_{\alpha,\beta=1}^2A^{\alpha\beta}\pder{}{x^\alpha}\pder{}{x^\beta}+\sum_{\alpha=1}^2B^\alpha\pder{}{x^\alpha}+C
\end{align*}
and $f$ is some forcing function. We will also enforce that $A$ is
a positive definite matrix. And where $u$ is a function that takes the position
and the time and returns a real $u(\vec{x},t):\R^2\times\R\rightarrow\R$. Note
that $\vec{x}\in\Omega$, and $t\in\R$.

Rewriting this expression into a single equation we can express our problem
statement as
\begin{align}\label{eq:prob}
  \pder{u}{t}=
  \sum_{\alpha,\beta=1}^2A^{ij}\pder{}{x^\alpha}
  \left[\pder{u}{x^\beta}\right]
  +\sum_{\gamma=1}^2B^\gamma\pder{u}{x^\gamma}+Cu+f
\end{align}

Note that for this process to function, we must also require some conditions.
We first need the boundary conditions in order to determine any solution.
Therefor we require that $u(\partial\Omega, t)$ is provided. Note that this
could mean that the boundary is split into several sections, each one with
different conditions, but the value of $u$ at any point along the boundary must
be known.

If considering the more general time dependent form of the equation we are also
required to know initial conditions. That is to say that $u(\Omega,0)$ must
also be provided. Using the initial conditions and the boundary conditions, we
are able to construct an approximation through the use of finite element
methods.

Given $A$, $B$, $C$, $f$, and appropriate conditions, we must construct
a solution for $u$. This is the key purpose of finite element methods, as for
most circumstances there does not exists an analytical solution to this
differential equation. Thus we must utilize finite element analysis in order to
construct the best possible approximation of the solution to the differential
equation, and then it is possible to use that approximation for further
analysis.

\end{document}
