\documentclass[../fem.tex]{subfile}

\begin{document}
\section{Assembly Of The Global System}%
\label{sec:assembly_of_the_global_system}

Although it is possible to directly compute the elements of the global
matrices, it is ineffective and not practical for any non trivial problems.
Thus we will not consider the explanation of how to construct the global
matrices directly.

The method that we utilize is an incremental construction of the global matrix,
through the addition of elements from each of the local matrices. We begin by
initializing the matrix $G$, $M=\wb{A}+\wb{B}+\wb{C}$ and $F$ of zeros. $G$ and
$M$ are $N\times N$ matrices, and $F$ is a $N\times 1$ matrix, where $N$ is the
number of vertices in the triangulation.

Every element matrix $G^\e$ and $M^\e$ are $3\times 3$ matrices, and each element matrix
$F^\e$ is a $3\times 1$ matrix. We need to construct a method that will allow
us to associate the elements of the local matrix to the elements of the global
matrix. The first part of this is determining a method to associate the element
vertex to the global vertex. We define a function $node$ to do just this
\begin{align*}
  node&:\R\times\{1,2,3\}\rightarrow\R\\
  node(e,I)&=i.
\end{align*}
This construction of the $node$ function, is such that given a element index,
and the local index of a vertex in the specified element, it will return the
global vertex number.

Using this function, it is possible to construct the algorithm to assemble the
global system of equations. The general method is to add each local matrix
element to the associated global matrix element, where the association is done
using the $node$ function. This can be viewed as
\begin{align*}
  A_{node(e,I)node(e,J)}=A_{node(e,I)node(e,J)}+A^\e_{IJ}\quad I,J=1,2,3.
\end{align*}
This is done for all of the elements in the mesh, and all of the local matrix
elements for each mesh element.

More specifics of the algorithm for the global matrix construction are
discussed in section \ref{sec:2assbemly_of_the_global_system}.

For the time independent problem statements, this is the extent of what needs
to be done for the assembly of the global system of equations. It provides us
with an expresion of the form
\begin{align*}
   MU=F.
\end{align*}

So far in the construction we have neglected the boundary basis functions, and
that causes this matrix to be unsolvable. Fixing this is the next step of the
process.
\end{document}
