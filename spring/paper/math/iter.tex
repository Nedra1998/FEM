\documentclass[../fem.tex]{subfiles}

\begin{document}
For the time dependent expressions, we need to be able to solve the system of
equations imperatively. For this one can use any number of methods, such as
explicit Euler, implicit Euler, Runge-Kutta, or Crank-Nicolson. For now we
will use the Crank-Nicolson method of iteration, as it provides significantly
more accurate results than either Euler method, and is significantly more
simplistic compared to Runge-Kutta.

\subsection{Euler}%
\label{sub:euler}

For the time dependent iteration, it is important to note that both $U$ and $F$
will be dependent on the time, or in this case the current iteration. These are
the only matrices that are dependent on the time, because in our expressions
for the evaluation of the elements of the matrices (equation
\ref{eq:local_sys}), the only elements that have a time dependent function is
that of $F$, thus $F$ is the only input matrix that is dependent on time. And
it is clear that if $F$ is dependent on the iteration, that $U$ will also be
dependent on the iteration.

Because of this dependence on the iteration, we will denote the current
iteration by using a superscript $n$. Thus we can rewrite $U$ and $F$ as $U^n$
and $F^n$ respectively.

The basis of the time iteration, is to use the approximation
\begin{align*}
  \pder{U}{t}\approx\frac{U^{n+1}-U^n}{\Delta t}.
\end{align*}
This approximation is directly derived from the definition of a derivative, and
it is clear that as $\Delta t\rightarrow 0$ then the approximation becomes an
equality.

Substituting this approximation into the matrix formulation of the system of
equations, we find the new expression to become
\begin{align*}
  G\frac{U^{n+1}-U^n}{\Delta t}=MU^n+F^n.
\end{align*}
Now we solve this equation for $U^{n+1}$ as it is safe to assume that $U^n$ is
already known from the previous iteration. Thus the expression
\begin{align*}
  GU^{n+1}&=\left(MU^n+F^n\right)\Delta t+GU^n\\
  GU^{n+1}&=\left(G+M\Delta t\right)U^n+F^n\Delta t
\end{align*}
is found.

Using this expression, it becomes simple to compute the value of $U^{n+1}$.
This is because, we can further rewrite this expression as
\begin{align*}
  GU^{n+1}&=Q\\\text{where}\ Q&=\left(G+M\Delta t\right)U^n+F^n\Delta t.
\end{align*}
It is well known how to solve a system of equations of the form $Ax=b$, so this
form of the expression can be solve for the value of $U^{n+1}$.

Unfortunately the Euler method only provides first order accuracy, so we will
instead utilize the Crank-Nicolson method of iteration.

\subsection{Crank-Nicolson}%
\label{sub:crank_nicolson}

For the Crank-Nicolson method of iteration, we follow a similar process to the
Euler method of iteration. The main difference is that instead of using the old
values of $U$ and $F$ on the right hand side of the expression, one uses the
average of the old and the new values. This is written as
\begin{align*}
  G\frac{U^{n+1}-U^n}{\Delta t}=M\frac{U^{n+1}+U^n}{2}+\frac{F^{n+1}+F^n}{2}.
\end{align*}

Once again, we assume that $U^n$ is known from the previous iterations, and
$F^{n+1}$, and $F^n$ can be computed as givens. Thus we solve for $U^{n+1}$. We
find that the expression becomes
\begin{align*}
  \left(G+\frac{1}{2}M\Delta t\right)U^{n+1}=\left(G-\frac{1}{2}M\Delta
  t\right)U^n+\frac{F^{n+1}+F^n}{2}\Delta t
\end{align*}

In order to simplify this expression, we define three new matrices to be
\begin{align*}
  \wt{A}&=G+\frac{1}{2}M\Delta t\\
  \wt{B}&=G-\frac{1}{2}M\Delta t\\
  \wt{C}^n&=\frac{F^{n+1}+F^n}{2}\Delta t.
\end{align*}

Now the expression be be rewritten as
\begin{align*}
  \wt{A}U^{n+1}=\wt{B}U^n+\wt{C}^n.
\end{align*}
Since $U^n$ is assumed to be known from the previous iteration, we can define a
new matrix $Q^n$ to be
\begin{align*}
  Q^n=\wt{B}U^n+\wt{C}^n.
\end{align*}
Using this new matrix, we can rewrite the expression one final time to be
\begin{align*}
  \wt{A}U^{n+1}=Q^n.
\end{align*}
This is finally in the format that is well known how to solve for the values of
$U^{n+1}$, so all that needs to be done is apply some algorithm to solve this
linear system.
\end{document}
