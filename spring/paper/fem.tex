\documentclass[12pt]{armath}
\usepackage{amsmath}
\usepackage{amsthm}
\usepackage{csquotes}
\usepackage{enumitem}

\usepackage[margin=1in]{geometry}

\usepackage{amsthm}
\usepackage{multicol}
\usepackage{tikz}
\usepackage{pgfplots}
\usepackage{subfiles}
\usepackage{minted}
\usepackage{algpseudocode}
\usepackage{algorithm}
\usepackage{graphics}
% \usepackage{physics}

\title{Finite Element Analysis}
\author{Arden Rasmussen}
\date{\today}

\numberwithin{equation}{section}

\newenvironment{Figure}
{\par\medskip\noindent\minipage{\linewidth}}
{\endminipage\par\medskip}

\theoremstyle{definition}
\newtheorem{example}{Example}[section]
\newtheorem{definition}{Definition}[section]

\renewcommand{\thealgorithm}{\arabic{section}.\arabic{algorithm}}
\renewcommand\thefigure{\thesection.\arabic{figure}}

\newcommand{\N}{\mathbb{N}}
\newcommand{\R}{\mathbb{R}}
\newcommand{\K}{\mathcal{K}}
\newcommand{\ep}{{\varepsilon}}

\newcommand{\pder}[2]{\frac{\partial #1}{\partial #2}}
\newcommand{\px}{{\partial x}}
\newcommand{\ph}{\varphi}
\newcommand{\e}{{(e)}}
\newcommand{\wt}[1]{\widetilde{#1}}
\newcommand{\wb}[1]{\overline{#1}}
\newcommand{\braket}[2]{\left<{#1},{#2}\right>}

\newcommand{\hdiv}[3]{
  \vspace{#1}%
  \noindent\rule{\textwidth}{#2}%
  \vspace{#3}%
}

\begin{document}
\maketitle

\begin{abstract}
  Finite Element Analysis (FEA) is the computational process prescribed by
  Finite Element Methods (FEM) in order to solve for numerical solutions to
  boundary value problems for partial differential equations. Finite Element
  Analysis is used in many specializations of mechanical engineering, and is
  used frequently in that discipline, and occasionally in other disciplines as
  well. This paper provides an introduction to the mathematics of Finite
  Element Methods, and why they are useful. Then, it provides and explains the
  simplistic implementation of Finite Element Analysis. The implementation
  provided with this paper is written in C++, and is commented for better
  readability of the code.
\end{abstract}

% \begin{multicols}{2}
\subfile{math/intro}
% \end{multicols}

\tableofcontents

\newpage
\hdiv{10pt}{0.5pt}{1pt}
\part{Mathematics}%
\label{prt:mathematics}
\hdiv{1pt}{0.5pt}{10pt}

% \begin{multicols}{2}

\subfile{math/prob}
\subfile{math/cond}
\subfile{math/vari}
\subfile{math/disc}
\subfile{math/local}
\subfile{math/mesh}
\subfile{math/global}
\subfile{math/bndry}
\subfile{math/lin}
\subfile{math/approx}

\newpage
\hdiv{10pt}{0.5pt}{1pt}
\part{Computational}%
\label{prt:computational}
\hdiv{1pt}{0.5pt}{10pt}

\newpage
\nocite{*}
\bibliographystyle{alpha}
\bibliography{references}
% \end{multicols}
\end{document}
