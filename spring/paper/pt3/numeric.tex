\documentclass[../fem.tex]{subfile}

\begin{document}
\section{Numeric Approximations}%
\label{sec:numeric_approximations}

For the method of finite elements, we require the computation of a number of
approximations. Most of these approximations are inherent in the method, and
cannot be avoided. However, there are some computations that are possible to do
by hand, but for the structure of the program, we chose to implement these
computationally.

These computations primarily include the integration of some function over a
triangular element, and the partial differentiation of some functions. These
computations are required in the stage to construct the matrix. This process is
demonstrated in section \ref{sec:assembly_of_the_global_system}, and the
equations that require the computations are shown in equation
\ref{eq:local_sys}.

\subsection{Local Basis Differentiation}%
\label{sub:local_basis_differentiation}

For the local basis function, it is possible to construct the partial
derivatives analytically. This is because the structure of the local basis
functions is constant, and known. Because of this we determined the partial
derivatives of the local basis functions to be
\begin{align}\label{eq:local_basis_deriv}
  \begin{split}
    \pder{\ph^\e_1}{x}&=\frac{\left(Y^\e_2-Y^\e_3\right)}{\left(Y^\e_2-Y^\e_3\right)\left(X^\e_1-X^\e_3\right)+\left(X^\e_3-X^\e_2\right)\left(Y^\e_1-Y^\e_3\right)}\\
    \pder{\ph^\e_1}{y}&=\frac{\left(X^\e_3-X^\e_2\right)}{\left(Y^\e_2-Y^\e_3\right)\left(X^\e_1-X^\e_3\right)+\left(X^\e_3-X^\e_2\right)\left(Y^\e_1-Y^\e_3\right)}.
  \end{split}
\end{align}
Because it is possible to only utilize the first local basis functions, this is
all that we need to compute. This is because, we are always able to rotate the
local basis functions. The actual number are unimportant, only the order of the
sequence matters.

Using this, we are able to use the actual value of the partial derivatives of
the local basis functions in the computation. Thus eliminating one of the
necessary approximations.

\subsection{Differentiation}%
\label{sub:differentiation}

Unfortunately in the computation of $\wb{A}^\e_{ij}$ we are required to take the
derivative of some user defined function. For the sake of not requiring the
user to manually compute the partial derivatives, and define those functions as
well, we chose to numerically approximation the partial derivative of this
arbitrary function.

Consider a function $f(x,y)$. The definition of a derivative is
\begin{align*}
  \pder{f}{x}\equiv\lim_{h\rightarrow 0}\frac{f(x+h,y)-f(x,y)}{h}.
\end{align*}
this equation is a simple two point estimation, where taking two points near
by, and then the difference should be the approximation of the derivative.
However, this approximation is crude, and does not provide much accuracy. One
with minor improvements is the symmetric difference quotient
\begin{align*}
  \frac{f(x+h,y)-f(x-h,y)}{2h}.
\end{align*}
This expression improves slightly on the accuracy of the derivative.

For our purposes, we want the numerical approximations of the derivatives to be
as accurate as possible. This is why we chose to use finite differences to
approximation the derivatives. These methods are the same as the two
expressions provided, just at a larger scale. The two expressions previously
presented are two point approximations of the derivative, but it is possible to
make a $k$-point approximation of the derivative. As $k$ increases the accuracy
of the derivative also increases.

For our implementation we chose to utilize an 8-point approximation of the
derivative. The coefficients of the approximation utilized were determined by
\cite{FD}. The approximation that we implemented takes the form
\todo{HELP}
\begin{align*}
  \resizebox{\columnwidth}{!}{$
    \pder{f}{x}\Bigr|_{(x,y)}\approx\frac{\frac{1}{280}f(x-4h,y)-\frac{4}{105}f(x-3h,y)+\frac{1}{5}f(x-2h,y)-\frac{4}{5}f(x-h,y)+\frac{4}{5}f(x+h,y)-\frac{1}{5}f(x+2h,y)+\frac{4}{105}f(x+3h,y)-\frac{1}{280}f(x+4h,y)}{h}+O\left(h^8\right).
$}
\end{align*}
Using this approximation, we achieve an error of $O\left(h^8\right)$. For many
cases this should be an acceptable amount of error.

Since each derivative is only done with respect to a single element, then we
are able to construct a more specific approximation. Currently one must
minimize the value of $h$ in order to compute more accurate approximations, but
for practical purposes, we must determine a value of $h$ to evaluate at. It is
not possible to evaluate this approximation with $h=0$. In order to determine
the value of $h$ to utilize, we base it off of the size of the element.

We denote the area or the measure of an element as $\mu\left(E_e\right)$. This
can be computed by the use of the following expression
\begin{align*}
  \mu\left(E_e\right)=\frac{1}{2}\left\vert
  X^\e_1\left(Y^\e_2-Y^\e_3\right)+X^\e_2\left(Y^\e_3-Y^\e_1\right)+X^\e_3\left(Y^\e_1-Y^\e_2\right)\right\vert.
\end{align*}
We decided to use $h=\frac{1}{8}\mu\left(E_e\right)$ as the value of $h$ in the
approximation of the partial derivatives. If for any reason this is not
sufficient for any problem, then all that needs to be done is to refine the
mesh in that region of the domain. This mesh refinement can be controlled by
the user through the mesh grading function (Def \ref{def:well_sized}).

\subsection{Integration}%
\label{sub:integration}

The final unavoidable approximation is the integration of arbitrary functions
over a triangular element. Due to the triangular nature of the elements, it is
not possible to utilize a straightforward implementation of Simpson's method, or
a trapezoidal rule. In order to integrate on an arbitrary triangular element we
must do several steps. Consider a function $f$ to be integrated over the
element $E_e$. Thus we wish to construct an approximation for
\begin{align*}
  \int_{E_e}fdA.
\end{align*}

The first step is to convert the arbitrary triangle into a master triangle
element. We define the master element, to be a triangular element such that the
vertices are located at $(0,0)$, $(1,0)$, and $(0,1)$, we denote this master
element as $\Delta$. This conversion induces a scale factor of the area of the
triangular element. Thus our expression becomes
\begin{align*}
  \int_{\Delta}f\mu\left(E_e\right)dA=\mu\left(E_e\right)\int_{\Delta}fdA.
\end{align*}

At this stage it would be possible to utilize Simpson's method or a trapezoidal
method for integration, changing the bounds of the $y$ integration based upon
the $x$ value, and the expression would be of the form
\begin{align*}
  \mu\left(E_e\right)\int_{x=0}^{x=1}\int_{y=0}^{y=1-x}fdydx.
\end{align*}
However, this method is extremely computationally intensive and does not
provide very accurate results for the computational complexity that it
requires. To compute this integral, we will utilize a different method of
numerical integration called Gaussian Quadrature.

The Gaussian Quadrature rule is a quadrature rule that has been constructed
such that a $n$-point Gaussian quadrature will yield an exact result for
polynomials of the degree $2n-1$. A quadrature rule is most methods of numeric
integration, for example trapezoidal rule is a quadrature rule. The premise of
the Gaussian quadrature rule is that the sum of a function evaluated at a set
of selected points multiplies by selected weights is able to compute the
approximation of the function.

This is the same concept as the trapezoidal rule, which is expressed as
\begin{align*}
  \int_a^bfdx\approx\frac{\Delta
    x}{2}\sum_{k=1}^{N}\left(f(x_{k-1})+f(x_{k})\right).
\end{align*}
This method is clearly a sum of the function as specified points $x_k$, multiplied by
a specified weight $\frac{\Delta x}{2}$.

Gaussian quadrature rules have been developed for the master triangle $\Delta$,
so we will utilize the computed weights and points from \cite{GQ}. This means
that the approximation of the integral becomes
\begin{align*}
  \mu\left(E_e\right)\int_{\Delta}fdA\approx\mu\left(E_e\right)\sum_{k=1}^Nw_kf(x_k,y_k).
\end{align*}
For our purposes we chose $N$ to be 128, as that will be sufficiently large that
most error should be negligible. If the error is not significantly negligible,
then the mesh should be refined thus reducing $\mu\left(E_e\right)$, and
consequentially reducing the scale of the error. The values from \cite{GQ} have
been provided in the appendix, both of the weights and the $(x,y)$ pairs.
\todo{Add weights and points in the appendix.}

\end{document}
