\documentclass[../fem.tex]{subfile}

\begin{document}
\section{Plotting}%
\label{sec:plotting}

For the plotting of the approximations generated by the finite element methods,
we utilize the global basis functions. The method that we utilize, is to
iterate through every pixel of the desired image, and then determine the
appropriate color. The first step is to determine which triangular element the
point is within. Using the element and the point, we are able to evaluate the
approximation at that point.

Once all of the values for every point in the grid have been stored, we then
determine the minimum and the maximum value. Then the values are mapped
$[min,max]\rightarrow[0,255]$. Using the mapped values we select a color from
an array of colors. The selected color is set as the color of the pixel at the
point.

We then utilize libpng\cite{PNG}, to save the image buffer data to a file. This
process can be slow, so we implement the images saving in a separate thread.
This is important so that more computations can be done on the main thread.

For the time dependent methods, we save multiple images into a folder, and use
FFMPEG\cite{FFMPEG} to construct an animation from the sequence of images.

\end{document}
