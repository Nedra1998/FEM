\documentclass[../fem.tex]{subfile}

\begin{document}
\section{Basis Functions}%
\label{sec:basis_functions}

Using the basis functions that were mathematically constructed in section
\ref{sec:localization}, we will now construct a method for the implementation
of these basis elements. Since these basis functions are used frequently and
commonly throughout the computations, it will be important to optimize the
calculations.

We will need both the local and global basis functions. The local basis
functions are utilized in the matrix construction from section
\ref{sec:assembly_of_the_global_system}, and the global basis functions are
utilized in the plotting and post processing from section \ref{sec:plotting}.
First we will concentrate on global basis functions, then will consider the
global basis functions.

\subsection{Local Basis Functions}%
\label{sub:local_basis_functions}

Recall that the local basis functions were derived in section
\ref{sub:local_basis}. The basis functions can be written as
\begin{align*}
  \ph^\e_1&=\frac{(Y^\e_2-Y^\e_3)(x-X^\e_3)+(X^\e_3-X^\e_2)(y-Y^\e_3)}{(Y^\e_2-Y^\e_3)(X^\e_1-X^\e_3)+(X^\e_3-X^\e_2)(Y^\e_1-Y^\e_3)}\\
  \ph^\e_2&=\frac{(Y^\e_3-Y^\e_1)(x-X^\e_3)+(X^\e_1-X^\e_3)(y-Y^\e_3)}{(Y^\e_2-Y^\e_3)(X^\e_1-X^\e_3)+(X^\e_3-X^\e_2)(Y^\e_1-Y^\e_3)}\\
  \ph^\e_3&=1-\ph^\e_1-\ph^\e_2.
\end{align*}
Clearly these basis functions will require the mesh, the element index, and the
local vertex index.

The first optimization is done with relation to $\ph^\e_3$.  Currently to
compute $\ph^\e_3$ one must also compute $\ph^\e_1$ and $\ph^\e_2$. However,
since the base index of the local indicies does not matter, only the ordering
of them, it is possible to only compute the value of $\ph^\e_1$. Since for
every other local index, we can consider rotating the ordering of the local
indicies, thus essentially converting the local index of $2$ or $3$ to $1$.
This process is demonstrated in figure \ref{fig:local_rotate}.

\end{document}
