\documentclass[../fem.tex]{subfile}

\begin{document}
\section{Incremental Construction}%
\label{sec:incremental_construction}

The algorithm that we implement for the construction of the global matrix,
closely follows the methods that are outlined in section
\ref{sec:assembly_of_the_global_system}.

\begin{algorithm}[H]
  \caption{element-by-element assembly}
  \begin{algorithmic}
    \State{Let $G$ by a $N\times N$ matrix of zeros.}
    \State{Let $M$ by a $N\times N$ matrix of zeros.}
    \State{Let $F$ by a $N\times 1$ matrix of zeros.}
    \ForAll{$E_e$}
    \State{Let $G^\e$ be a $3\times3$ matrix, filled with values from
    integration.}
    \State{Let $M^\e$ be a $3\times3$ matrix, filled with values from
    integration.}
    \State{Let $F^\e$ be a $3\times1$ matrix, filled with values from
    integration.}
      \State{$i=node(e,I)\quad j=node(e,J)\quad I,J=1,2,3$}
      \State{$G_{ij}=G_{ij}+G_{IJ}^\e$}
      \State{$M_{ij}=M_{ij}+M_{IJ}^\e$}
      \State{$F_{i}=F_{i}+F_{I}^\e$}
    \EndFor
    \State{\Return $G$, $M$ and $F$.}
  \end{algorithmic}
\end{algorithm}

Where $node(e,I)$ provides the relation between global and local node numbers.
Such that given an element and element vertex number, it returns the global
vertex number. With the triangular mesh, there is no nice mathematical formula
for this expression, but it is simple for any implementation.

\end{document}
