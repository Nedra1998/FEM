\documentclass[../fem.tex]{subfile}

\begin{document}
\section{Imposing Boundary Conditions}%
\label{sec:imposing_boundary_conditions}

So far in the process of the construction of the global system of equations, we
have not taken into consideration the conditions on the boundary. As we
introduced previously, we will first construct the approximation using the
basis elements that are supported only on the interior of the domain. However,
the process so far has constructed the matrices, including the basis elements
that are on the boundary. At this stage, we need to remove these basis elements
from the approximation. We can do this in conjunction with the construction of
the $b$ function, which satisfies the boundary conditions.

First, we will deal with removing the basis function on the boundary of the
domain. We consider the general form of the matrix expression that we will be
considering to be
\begin{align*}
   Au=b.
\end{align*}
We write this expression explicitly as
\begin{align*}
   \begin{bmatrix}
     A_{11} & A_{12} & \cdots & A_{1N}\\
     A_{21} & A_{22} & \cdots & A_{2N}\\\
     \vdots & \vdots & \ddots & \vdots\\
     A_{N1} & A_{N2} & \cdots & A_{NN}
   \end{bmatrix}
   \begin{bmatrix}
      u_1\\u_2\\\vdots\\u_N
   \end{bmatrix}=
   \begin{bmatrix}
      b_1\\b_2\\\vdots\\b_N
   \end{bmatrix}.
\end{align*}

As we know what the value of the function is defined as at the boundary
conditions, then we also are able to determine what the value of the function
is at any vertex of the mesh that lies on the boundary.

Consider a vertex $\vec{x}_i$ on the boundary of the mesh.  We modify the
system of equations by overwriting the equations associated with the $u_i$
value, to satisfy the conditions. This can be done by use of the expressions
\begin{align*}
  A_{ij}&=\delta_{ij}\quad j=1,\ldots,N\\
  b_i&=\partial u(\vec{x}_i)
\end{align*}
where $\delta_{ij}$ is the Kronecker delta function
\begin{align*}
  \delta_{ij}=
  \begin{cases}
    0 & j\neq i\\
    1 & j= 0
  \end{cases}.
\end{align*}

This modification must be performed for all vertices of the mesh that lie on
the boundary of the domain. This process will present a matrix expression that
will be of
the form
\begin{align*}
   \begin{bmatrix}
     A_{11} & A_{12} & \cdots & A_{1N}\\
     0 & 1 & \cdots & 0\\\
     \vdots & \vdots & \ddots & \vdots\\
     A_{N1} & A_{N2} & \cdots & A_{NN}
   \end{bmatrix}
   \begin{bmatrix}
      u_1\\u_2\\\vdots\\u_N
   \end{bmatrix}=
   \begin{bmatrix}
     b_1\\\partial u(\vec{x}_2)\\\vdots\\b_N
   \end{bmatrix}.
\end{align*}
This example demonstrates a matrix where the second vertex lies on the
boundary, and so that equation has been rewritten. The same process would have
been done for any other vertex that lies on the boundary.

This process has enforced the requirements of the boundary conditions, by
forcing $u_2=\partial u(\vec{x}_2)$. This process incorporates the $b$ function
directly into the approximation of the solution, so we need not worry about
that.

\end{document}
