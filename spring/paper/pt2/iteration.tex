\documentclass[../fem.tex]{subfile}

\begin{document}
\section{Temporal Iterations}%
\label{sec:temporal_iterations}

For further iteration of the solution, to solve for the solution at later
times, we utilize the temporal discretization constructed in section
\ref{sec:temporal_discretization}.

The first stage of this is to compute the next forcing term $F^{n+1}$, and the
current forcing term $F^{n}$. For our cases, this should just be dependent on
time, and position, so that can easily be computed using the same methods used
to compute the initial forcing term.

Then we construct the $\wt{C}^{n}$ matrix. Note that we need not reconstruct
$\wt{A}$ or $\wt{B}$, since those are independent of the current time step, and
so can be computed once and used for all iterations. We construct
$\wt{C}^{n}$ by
\begin{align*}
  \wt{C}^{n}=\Delta t\frac{F^{n+1}+F^{n}}{2}.
\end{align*}

Then using the values of $\wt{C}^n$ and $U^n$ we construct the expression for
$Q^n$ to be
\begin{align*}
  Q^n=\wt{B}U^n+\wt{C}^n.
\end{align*}
Then we can use these expressions to solve for the solution coefficients at the
next time step, by the expression
\begin{align*}
  \wt{A}U^{n+1}=Q^n.
\end{align*}

This process is more closely outlined in the explanation of the Crank-Nicolson
method in section \ref{sub:crank_nicolson}.


\end{document}
