\documentclass[../fem.tex]{subfile}

\begin{document}
\section{Construction of the Approximation}%
\label{sec:construction_of_the_approximation}

Once the linear system is solved, we are provided with a vector of values $U^n$.
However, we need to develop a method for constructing the approximation from
this vector of values. Since these values should be the coefficients of the
linear combination of the global basis functions $\ph_i$, then we construct the
approximation $u(n\Delta t, x)$ to be
\begin{align*}
  u\left(x, n\Delta t\right)=\sum_{i=1}^NU^n_i\ph_i.
\end{align*}

Recall that $n$ is the current time step, and $\Delta t$ is the size of each
time step, and that $N$ is the number of vertices in the mesh.

This is the expression for the approximation at a single given time of
$t=n\Delta t$. For the solution at earlier times, one must look at previously
computed time steps where the solution has already been computed. For solutions
at later times, we need to compute further iterations.

Using this method of time iterations, our approximation is only fully defined
at the time values of $n\Delta t$, this means that if one wants a solution at a
time value of $\frac{1}{2}n\Delta t$, then we are unable to provide a solution.
To work around this, we can consider that the solutions are continuous
temporally, and so if the time steps are small enough, we can consider the
solution to change linearly with time between each time step. This means that
we are able to construct a solution at any time, if we are given the
coefficients in the form of the $U^n$ and $U^{n+1}$ vectors.

\end{document}
