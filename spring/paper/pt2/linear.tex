\documentclass[../fem.tex]{subfile}

\begin{document}
\section{Linear Systems}%
\label{sec:linear_systems}

Using finite element methods, constructs a system of equations on the order of
$N\times N$, where $N$ is the number of verticies in the triangulation. With
finer meshes will produce a finer approximation of the solution to the partial
differential equation. Because of this, most implementation will have highly
refined meshes, ones that would have $\sim 10^4$ verticies. Because of this one
must develop a more efficient method for computing solutions to systems of
linear equations. Most is this process is discuessed further in the
section \ref{sec:linear_system_b}. However we will introduce some mathematical
concepts that are heavily used in the computational methods.

\subsection{Krylov Subspace}%
\label{sub:krylov_subspace}

An extreamly useful tool for numerical approximation of large systems of
equations is the \textit{Krylov Subspaces}. Krylov subspaces are defined by
Krylov \cite{KRYLOV} as
\begin{align*}
  \K_r(A,b)=\text{span}\left\{b,Ab,A^2b,\ldots,A^{r-1}b\right\}
\end{align*}
where $b$ is some vector. This basis assists in the approximation for linar
systems of the form $Ax=b$. The subscript $r$ simply defines the point at which
we limit our approximation, if $r$ is infinite then the Krylov subspace is the
full space, ans dolutions in that krylov subspace will be exact.

Note that the basis vector in the Krylov subspace are not neccessarily
orthogonal, nor normal.

\subsubsection{Reasoning}%
\label{ssub:reasoning}

We provide a sort example of the reasoning for the construction of the Krylov
subspace. We consider the example where we want to solve a system of linear
equations of the form $Ax=b$. Where $A=I+\ep$.
\begin{align*}
  Ax&=b\\
  x&=A^{-1}b\\
  x&=(I+\ep)^{-1}b.
\end{align*}

We construct an approximation for the invers of a matrix-like by geometruc
formula
\begin{align*}
  I+\ep+\ep^2+\ldots+\ep^n&=S\\
  \ep+\ep^2+\ldots+\ep^{n+1}&=W\ep\\
  S-I+\ep^{n+1}&=S\ep\\
  S-S\ep&=I-\ep^{n+1}\\
  S(I-\ep)&=I-\ep^{n+1}\\
  S&=\left(I-\ep^{n+1}\right)(I-\ep)^{-1}.
\end{align*}
When $n\rightarrow\infty$ and $\ep$ is small enough ($\Vert\ep\Vert<1$), then
$\ep^{n+1}\rightarrow0$. Thus
\begin{align*}
  S&=\left(I-\ep^{n+1}\right)(I-\ep)^{-1}\\
  S&=(I-\ep)^{-1}.
\end{align*}
Finally we conclude that
\begin{align*}
  A^{-1}=(I+\ep)^{-1}=I-\ep-\ep^2-\ldots.
\end{align*}
Using this in our expression for $x$ we find
\begin{align*}
  x&=A^{-1}b=b-\ep b-\ep^2b-\ldots\\
   &=b-(A-I)b-(A-I)^2b-\ldots\\
   &=b-Ab+b-A^2b+2Ab-b+\ldots.
\end{align*}
Using this we extract the basis that composes $x$ to be
\begin{align*}
  \left\{b,Ab,A^2b,A^3b,\ldots\right\}
\end{align*}

The Krylov subspaces takes this infinite basis and cuts it off at some given
$r$ values, thus providing us with
\begin{align*}
  \left\{b,Ab,A^2b,\ldots,A^{r-1}b\right\}.
\end{align*}
Which the the Krylov subspaces.

Note that this is not a full proof of the construction of the Krylov subspace,
but it shoulbe be a sufficient explanation for the uses in the numeric
approximation of systems of linear equations.

\end{document}
