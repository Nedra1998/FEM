\documentclass[../fem.tex]{subfile}

\begin{document}
\section{Temporal Discretization}%
\label{sec:temporal_discretization}

Since this problem is also dependent on time, we must not only discretize the
problem spatially, but also temporally.

Consider the expression constructed in the previous section (equation
\ref{eq:matrix_simp}). We will construct the temporal discretization of this
equation
\begin{align*}
   G\partial_t U=MU+F.
\end{align*}

We will first require some definition of what the time step is. The user must
define the size of the time step $\Delta t$, and the maximum time $\tm$. Now to
construct the discretization, we utilize notation that will denote what the
current time step is.

Take for example the solution $U$, we will denote the solution before any
temporal iteration to be $U^0$, and at the first time step to be $U^1$, and at
the $n$th time step to be $U^n$. Note that because the time steps are in
discrete steps of $\Delta t$, then the time at the $n$th time step is
$t=n\Delta t$. So we can consider $U^n$ to be the solution at $t=n\Delta t$.
This notation is used for several of the variables in the discretization. Using
this notation, we rewrite the expression to be
\begin{align}\label{eq:time_disc}
   G\partial_t U^n=MU^n+F^n.
\end{align}
Note that we did not place a time step index on $G$ or $M$. This is because in
the construction of the elements of the matrices in equation
\ref{eq:global_mat_ele}, it can clearly be seen that these matrices will be
independent of time. This is because we are restricting $A$, $B$, and $C$ to be
temporally constant, and the basis functions $\ph_i$ are are independent with
respect to time by construction.

There are many methods that can be used for the discretization of the problem.
We chose to utilize the Crank-Nicolson method, as it provides higher order
accuracy, while not costing significant amounts of computational power. For
significantly more accurate computation, one could implement a Runge Kutta
method for the temporal iteration of the problem.

The Crank-Nicolson method is a combination of the Forward Euler method, and the
Backward Euler method, so we provide an explanation of all three of these
methods below.

\subsection{Forward Euler Method}%
\label{sub:forward_euler_method}

The basis of this method is to consider the derivative as a difference
quotient, such as
\begin{align*}
  \partial_t U^n=\frac{U^{n+1}-U^n}{\Delta t}.
\end{align*}
Using this approximation in the equation \ref{eq:time_disc}, we find
\begin{align*}
  G\frac{U^{n+1}-U^n}{\Delta t}=MU^n+F^n.
\end{align*}
Now we solve this expression for the solution for the next time step, which
would be $U^{n+1}$. We find that the equation for $U^{n+1}$ becomes
\begin{align*}
  G\frac{U^{n+1}-U^n}{\Delta t}&=MU^n+F^n\\
  \frac{1}{\Delta t}GU^{n+1}&=MU^n+\frac{1}{\Delta t}GU^n+F^n\\
  GU^{n+1}&=\left(\Delta tM+G\right)U^n+\Delta tF^n.
\end{align*}
One can then define a matrix $Q^n$ to make this expression into the
recognizable form of $AU=F$, which can easily be solved. For this case
\begin{align*}
  Q^n\equiv\left(\Delta tM+G\right)U^n+\Delta tF^n.
\end{align*}

This is the forward Euler method for the approximation.

\subsection{Backward Euler Method}%
\label{sub:backward_euler_method}

The backward Euler method uses very similar methods to the forward method, with
one difference. The values on the right hand side of the expression for the
forward method, were all of the current time iteration $n$. The backward method
utilizes the updates values for the next time iteration $n+1$. Thus the
expression becomes
\begin{align*}
  G\frac{U^{n+1}-U^n}{\Delta t}=MU^{n+1}+F^{n+1}.
\end{align*}

We once again solve for the solution for the next time step
\begin{align*}
  G\frac{U^{n+1}-U^n}{\Delta t}&=MU^{n+1}+F^{n+1}\\
  GU^{n+1}-GU^n&=\Delta tMU^{n+1}+\Delta tF^n\\
  \left(G-\Delta tM\right)U^{n+1}&=GU^n+\Delta tF^n.
\end{align*}
Then similarly to the forward iteration, we construct a matrix $Q^n$ and a
matrix $\wt{A}$, to construct the form that we can numerically solve. For this
case
\begin{align*}
  \wt{A}&\equiv G-\Delta tM\\
  Q^n&\equiv GU^n+\Delta tF^n.
\end{align*}

This is the concept for the backward Euler method for the approximation.

\subsection{Crank-Nicolson}%
\label{sub:crank_nicolson}

The key concept of the Crank Nicolson method, is to combine both the forward
and backward Euler methods. That means that instead of using the old values or
the new values on the right hand side of the equation, we will use the average
of the old and the new values.

This means that the expression takes the form
\begin{align*}
  G\frac{U^{n+1}-U^n}{\Delta t}=M\frac{U^{n+1}+U^n}{2}+\frac{F^{n+1}+F^n}{2}.
\end{align*}

We now solve this expression for the solution to the next time step
\begin{align*}
  G\frac{U^{n+1}-U^n}{\Delta t}&=M\frac{U^{n+1}+U^n}{2}+\frac{F^{n+1}+F^n}{2}\\
  GU^{n+1}-GU^n&=\frac{\Delta t}{2}MU^{n+1}+\frac{\Delta t}{2}MU^n+\Delta
  t\frac{F^{n+1}+F^n}{2}\\
  \left(G+\frac{\Delta t}{2}M\right)U^{n+1}&=\left(G-\frac{\Delta
    t}{2}M\right)U^n+\Delta t\left(\frac{F^{n+1}+F^n}{2}\right).
\end{align*}
To simplify this expression we will define three new matrices to be
\begin{align*}
  \wt{A}&\equiv G+\frac{\Delta t}{2}M\\
  \wt{B}&\equiv G-\frac{\Delta t}{2}M\\
  \wt{C}^n&\equiv \Delta t\frac{F^{n+1}+F^n}{2}.
\end{align*}
Using these new matrices in the expression, we find that it simplifies to
\begin{align*}
  \wt{A}U^{n+1}=\wt{B}U^n+\wt{C}^n.
\end{align*}
For a final time, we will define a $Q^n$ matrix to construct the solvable form
of the expression. For our implementation we find
\begin{align*}
  Q^n\equiv\wt{B}U^n+\wt{C}^n.
\end{align*}

Thus the expression for the temporal approximation becomes
\begin{align*}
  \wt{A}U^{n+1}=Q^n.
\end{align*}
This is the expression for the temporal discretization of the problem, provided
by the Crank-Nicolson method. This is what will be used for any time dependent
expression to iterate through the time steps that are specified by the user.

\end{document}
