\documentclass[../fem.tex]{subfiles}

\begin{document}
\section{Dirichlet Boundary Conditions}%
\label{sec:dirichlet_boundary_conditions}

Now that the global system of equations has been constructed, we are able to
impose the Dirichlet boundary conditions to the system.

With our global system of equations in the general form of

\begin{align}\label{eq:global_sys}
  \begin{pmatrix}
    a_{11} & a_{12} & \cdots & a_{1N}\\
    a_{21} & a_{22} & \cdots & a_{2N}\\
    \vdots & \vdots & \ddots & \vdots \\
    a_{N1} & a_{N2} & \cdots & a_{NN}
  \end{pmatrix}
  \begin{pmatrix}
    T_1 \\ T_2 \\ \vdots \\ T_N
  \end{pmatrix}
  =
  \begin{pmatrix}
    F_1 \\ F_2 \\ \vdots \\ F_N
  \end{pmatrix}.
\end{align}

This system of equations is not well behaved, as we have neglected to prescribe
the boundary conditions until now. This means that our matrix is overdetermined
and cannot be solved. In order to fix this we must apply the boundary
conditions. By the Dirichlet boundary conditions we know that for any vertex in
the mesh that is on the boundary $\vec{x_i}\in\Gamma$,
\begin{align*}
  T_i=T_0(\vec{x_i}).
\end{align*}

We modify the system of equations by overwriting the equations associated with
the $T_i$ value, to satisfy the conditions. This can be done like so
\begin{align*}
  a_{ij}&=\delta_{ij}\quad j=1,\ldots,N\\
  F_i&=T_0(\vec{x_i}),
\end{align*}
where $\delta$ is the Kronecker delta function
\begin{align*}
  \delta_{ij}=\begin{cases}
    0 & j\neq i\\
    1 & j = i
  \end{cases}.
\end{align*}

This modification must be preformed for each vertex of the triangulation that
lies on the boundary. This process will present a matrix vaguely of the form
\begin{align*}
  \resizebox{\columnwidth}{!}{$
  \begin{pmatrix}
    a_{11} & a_{12} & \cdots & a_{1(N-1)}& a_{1N}\\
    0 & 1 & \cdots & 0 & 0\\
    a_{31} & a_{32} & \cdots & a_{3(N-1)} & a_{3N}\\
    \vdots & \vdots & \ddots & \vdots & \vdots\\
    0 & 0 & \cdots & 1 & 0\\
    a_{N1} & a_{N2} & \cdots & a_{N(N-1)} & a_{NN}
  \end{pmatrix}
  \begin{pmatrix}
    T_1 \\ T_2 \\ T_3 \\ \vdots \\ T_{N-1} \\ T_N
  \end{pmatrix}
  =
  \begin{pmatrix}
    F_1 \\ T_2 \\ F_3 \\ \vdots \\ T_{N-1} \\ F_N
\end{pmatrix}.$}
\end{align*}
This example shows a matrix where the $1$, and $N-1$ vertices lie on the
boundary, and so those equations are rewritten.

\end{document}
