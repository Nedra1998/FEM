\documentclass[../fem.tex]{subfiles}

\begin{document}
\section{Assembly of the Global System}%
\label{sec:assembly_of_the_global_system}

It is possible to directly construct the global matrix, however this method is
ineffective for practical computation and so in this paper we will ignore the
direct computation.

The method that we will use is an incremental construction of the global
matrix, through the addition of elements form the local matrices. We begin by
initializing the matrix $A=C+K$ and $F$ of zeros. $A$ is a $N\times N$ matrix,
and $F$ is a $N\times 1$ matrix, where $N$ is the number of vertices in the
triangulation. Then the corresponding contributions of the element matrices
$A^{(e)}$ and $F^{(e)}$ are added in a loop over all elements.

\begin{algorithm}[H]
  \caption{element-by-element assembly}
  \begin{algorithmic}
    \State{Let $A$ by a $N\times N$ matrix of zeros.}
    \State{Let $F$ by a $N\times 1$ matrix of zeros.}
    \ForAll{$E_e$}
      \State{Let $A^{(e)}$ be a $3\times3$ matrix, such that}
      \State{$a_{IJ}^{(e)}=c_{IJ}^{(e)}+k_{IJ}^{(e)}\quad I,J=1,2,3$}
      \State{$i=node(e,I)\quad j=node(e,J)\quad I,J=1,2,3$}
      \State{$A_{ij}=A_{ij}+A_{IJ}^{(e)}$}
      \State{$F_{i}=F_{i}+F_{I}^{(e)}$}
    \EndFor
    \State{\Return $A$ and $F$.}
  \end{algorithmic}
\end{algorithm}

Where $node(e,I)$ provides the relation between global and local node numbers.
Such that given an element and element vertex number, it returns the global
vertex number. With the triangular mesh, there is no nice mathematical formula
for this expression, but it is simple for any implementation.

\end{document}
