\documentclass[../fem.tex]{subfiles}

\begin{document}
\section{Variational Formulation}%
\label{sec:variational_formulation}

We first define the concept of a residual. The residual is the process of
constructing a helper function $R(f)$, such that when $f$ is the exact solution
$R(f)=0$. This simplifies the process of solving this partial differential
equation, into minimizing the value of the residual. This concept is used
multiple times throughout the process of FEA.

We construct the residual of equation \ref{eq:pde} as
\begin{align*}
  R(\vec{x})=\rho c_p\vec{u}\cdot\nabla T-\nabla\cdot(k\nabla T)-f.
\end{align*}
Clearly $R(\vec{x})=0$ if and only if $T$ is the exact solution. Multiplying
the residual by some test function $w$ and integrating, we obtain the weighted
residual
\begin{align*}
  \int_\Omega w(\vec{x})R(\vec{x})d\vec{x}.
\end{align*}

For the exact solution, this will always be zero because $R(\vec{x})=0$. Thus,
we can select any function for the test function $w$, and this expression can
still be satisfiable. However, we will restrict our test functions to ones that
are zero along the boundaries, this simplifies the computations later. We can
expand the weighted residual to obtain
\begin{align}\label{eq:weight_res}
  \int_\Omega\rho c_p w\vec{u}\cdot\nabla T d\vec{x} - \int_\Omega
  w\nabla\cdot(k\nabla T)d\vec{x}=\int_\Omega wf d\vec{x}.
\end{align}
Applying Green's theorem on the second integral we are able to express it as
\begin{align*}
  -\int_\Omega w\nabla&\cdot(k\nabla T)d\vec{x}\\
  &=\int_\Omega k\nabla w\cdot \nabla T d\vec{x} - \int_\Gamma kw\nabla T\cdot \vec{n}ds.
\end{align*}

The integral over $\Gamma$ goes to zero, because we require our test function
to be zero on the boundary. Thus we obtain the representation of the continuour
variational problem
\begin{align}\label{eq:cvp}
  \int_\Omega \rho c_p w \vec{u}\cdot\nabla Td\vec{x}+\int_\Omega k\nabla
  w\cdot\nabla Td\vec{x}=\int_\Omega wfd\vec{x}.
\end{align}


\end{document}
