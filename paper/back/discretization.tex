\documentclass[../fem.tex]{subfiles}

\begin{document}
\section{Discretization}%
\label{sec:discretization}

We define our approximate solution to be in the Soblev space $H^1(\Omega)$. The
Soblev space is a space of continuous functions that are infinitily
differentiable. The span of the Soblev space is infinite, so we will consider a
finite subspace $V_h\subset H^1(\Omega)$, and we will construct our
approximation in this subspace. We define a set of basis functions for the
subspace $V_h$ to be $\left\{\varphi_j\right\}$. We leave these basis functions
as arbatrairy functions for now, and will show the process for the
constructions of the basis functions in a later section.

The approximation of $T$ we define as $T_h$. It is posible to express this
apprximation as a linear combination of the basis functions $\varphi_j$ for the
finite subspace $V_h$ that $T_h$ is within. We write this expression as
\begin{align}\label{eq:lin_approx}
  T_h=\sum_{j=1}^NT_j\varphi_j(\vec{x}),
\end{align}
where $T_j$ is a constant that we solve for later to achieve our final
apprximation. We can now substitute our expression for $T_h$ for $T$ in the
continuous variational problem (\ref{eq:cvp}). Since the continuous variational
problem was satisfied for any test function, we make the decision to use
$\varphi_j$ as the test functions. These two substitutions provide us with the
expression
\begin{align*}
  \int_\Omega\rho c_p \varphi_i\vec{u}&\cdot\nabla\left[\sum_{j=1}^NT_j\varphi_j\right]d\vec{x}\\
  &+\int_\Omega k\nabla
  \varphi_i\cdot\nabla\left[\sum_{j=1}^NT_j\varphi_j\right]d\vec{x}
  =\int_\Omega \varphi_ifd\vec{x}.
\end{align*}
Which we then rewrite as
\begin{align*}
  \sum_{j=1}^N&\left[T_j\int_\Omega\rho c_p\varphi_i\vec{u}\cdot\nabla\varphi_jd\vec{x}\right]\\
  &+ \sum_{j=1}^N\left[T_j\int_\Omega
  k\nabla\varphi_i\cdot\nabla\varphi_jd\vec{x}\right]=\int_\Omega\varphi_ifd\vec{x}.
\end{align*}

This provides us with a system of equations for $i=1,2,\ldots,N$. Thus we
express this system in matrix notation as
\begin{align}\label{eq:mat_rep}
  (C+K)T=F.
\end{align}
Where $T$ is a vector in $\R^N$ of the $T_j$ constants, and the elements of the
matricies $C$, $K$, and $F$ are defined as
\begin{align*}
  c_{ij}&=\rho c_p\int_\Omega\varphi_i\vec{u}\cdot\nabla\varphi_jd\vec{x}\quad
        &i,j&=1,\ldots,N\\
  k_{ij}&=k\int_\Omega\nabla\varphi_i\cdot\nabla\varphi_jd\vec{x}\quad
        &i,j&=1,\ldots,N\\
  f_i&=\int_\Omega\varphi_ifd\vec{x}\quad&i&=1,\ldots,N.
\end{align*}

We note that is is often simpler to consider a matrix $A$ such that $A=C+K$,
then our expression for the system of equations becomes
\begin{align*}
   AT=F.
\end{align*}

This expression tends to be simpler to implement, and so we will continue by
using this expression. If other (non Dirichlet) boundary conditions had been
implemented, this expression for the system of equations would be different. As
the boundary conditions integrals are not zero, and so they will add an
additional term to this system of equations.

\end{document}
