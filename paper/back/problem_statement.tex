\documentclass[../fem.tex]{subfiles}

\begin{document}
\section{Problem Statement}%
\label{sec:problem_statement}

For our expanation we will consider the 2D steady state heat equations. This is
a simplification of the unsteady multidimensional heat transfer model. Since we
are focusing on the time independent analysis, there is no need for an initial
condition. We definine our domain to be $\Omega$ such that $\Omega \subset
\R^2$ is a bounded 2D domain with boundary $\Gamma$.

We write the general expression for the time-dependent general heat equation as
\begin{align*}
  \rho c_p \pder[T]{t}+\rho c_p\vec{u}\cdot\nabla T - \nabla
  \cdot\left(\bar{\bar{x}}\nabla T\right)=f \quad \text{in}\ \Omega.
\end{align*}
Where $T(\vec{x},t)$ is the temperature at point $\vec{x}\in\Omega$ at time
$t\in(0,\tau)$. $\rho$ is the density, and $c_p$ denotes the specific heat
capacity. For our case we assume that they are constant and $1$. The forcing
function $f$ is given by $f=\rho h$ where $h$ is the rate of heating per unit
mass. $\vec{u}$ is the velocity, and $\bar{\bar{x}}$ is the thermal
conductivity is a $2\times 2$ matrix.

Using many simplifications and our assumption that there is no change in time,
we find the problem statement to be
\begin{align}\label{eq:pde}
  \rho c_p\vec{u}\cdot \nabla T-\nabla\cdot(k\nabla T)=f \quad \text{in}\ \Omega.
\end{align}
Using the definition of gradient and divergence operators, we find that this
simplified equation can be written as
\begin{align*}
  \sum_{i=1}^2u_i\pder[T]{x_i}-\sum_{i=1}^2\sum_{j=1}^2\pder{x_i}\left(k\pder[T]{x_j}\right)=f
    \quad \text{in}\ \Omega.
\end{align*}

Note that the solution ot this equations is defined with some constant, meaning
that if $T$ is a solution to this equation, then $T+C$ for $C\in\R$ is also a
solution. Inorder to avoid this, we must impose some boundary conditions.

\end{document}
