\documentclass[../fem.tex]{subfiles}

\begin{document}
\section{Introduction}%
\label{sec:introduction}

Finite Element Methods (FEM) is a process that is utilized for
numerically computing an approximate solution to a partial differential
equations, on some specified domain. There are only a few cases where this
process is analytically solvable, but for most situations this approximation is
as close as technology currently allows us to achieve.

The uses of finite element methods in computation and analysis are often called
Finite Element Analysis (FEA), while the mathematics of the process is called
FEM.

We will utilize the process closely set forth by \cite{KH}. This method is
constructed around the Galerkin method for generating the approximation. The
scope of this introduction is limited to two dimensions, but all of the
mathematics is easily generalized to higher or lower dimensions. The only major
part that would require rework is Mesh Generation (Section
\ref{sec:mesh_generation}). The process can also be implemented for time-dependent systems, but we limit ourselves to time independence for our
introduction.

In the process of FEM that we analyze, there are two major sections,
\textbf{Theoretical}, and \textbf{Numerical}. The theoretical parts are
required to be completed prior to any numerical computation. These stages of
the process must be done manually in order to determine the program to
implement. These stages are extremely dependent on the general form of the
problem at hand and is where much of the mathematics is required. Then
there are the numerical computational stages. These steps are frequently
independent of the problem at hand, and thus much of the work can be automated.
A single implementation of the computational steps can be utilized for a
general class of problems. Aswell as many of the individual steps can be used
for any problem, and are completely independent of the rest of the process.
\end{document}
