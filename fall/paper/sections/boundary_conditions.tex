\documentclass[../fem.tex]{subfiles}

\begin{document}
\section{Boundary Conditions}%
\label{sec:boundary_conditions}

We must prescribe boundary conditions on our problem statement(Eq
\ref{eq:pde}), to ensure that there is a single solution. For this initial
introduction, we only present the process with Dirichlet boundary conditions
\begin{align*}
  T=T_0\quad\text{on}\ \Gamma,
\end{align*}
where $T_0(\vec{x})$ is some function defining the temperature along each of
the edges of the boundary. Note that this function can be of any form,
including a piecewise function.

It is important to note that other boundary conditions, such as Neumann, and
Newton boundary conditions can be used. These other boundary conditions add
more changes to the formulation of the problem later on and require extra
modifications to the implementation. Because of the extra alterations that they
would necessitate, we will only focus on the Dirichlet boundary conditions.

If the time-dependence of the problem had been retained, we would also require
some initial condition
\begin{align*}
  T(\vec{x}, 0) = T^0(\vec{x}).
\end{align*}
that would be defined on our domain $\Omega$. However, we are able to neglect
the initial condition as we are only looking for the stationary boundary value
problem solution.

\end{document}
