\documentclass[../fem.tex]{subfiles}

\begin{document}
\section{Conclusion}%
\label{sec:conclusion}

The process of Finite Element Analysis is extremely useful in practical
implementations for numerically solving partial differential equations on some
provided domain. The main stages of the process are to first determine the
continuous variational formulation of the partial differential equation, then
use that expression to construct the system of equations. We then generate a
mesh, and determine the system of equations for each individual element, and use
the element systems to modify the global system of equations. Then utilizing an
algorithm for approximating the solution to the system of linear equations, we
are then able to determine the final approximation of the system.

An advantage of this method is the amount of the process which can be
automated. The only need for human computation is for changing the differential
equation, the dimension of the problem (1D, 2D, or 3D), and for changing the
conditions at the boundaries (Dirichlet, Neuman, Newton, etc.). The rest of
the process can be completely handled by a computer. Because of this, many
variables such as the shape of the geometry, and the values at the boundaries
can be altered without any need for a new program.

Finite Element Methods is one of the industry leading methods for the
approximation of partial differential equations, and is frequently used in many
applications. This basic introduction to the process should provide an
the introductory level of knowledge on the mathematics, and the algorithms that are
utilized to implement finite element methods.

\end{document}
