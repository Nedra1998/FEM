\newif\ifdraft
% \drafttrue % Comment out to enable deployment

\ifdraft
\documentclass[draft,10pt]{amsart}
\else
\documentclass[10pt]{amsart}
\fi


\usepackage{amsfonts}
\usepackage{amsmath}
\usepackage{amsthm}
\usepackage{amscd}

\usepackage[margin=1in]{geometry}
\usepackage[obeyDraft]{todonotes}

\usepackage{csquotes}
\usepackage{enumitem}
\usepackage{multicol}
\usepackage{caption}
\usepackage{tikz}
\usepackage{pgfplots}
\usepackage{minted}
\usepackage{nameref}
\usepackage{algpseudocode}
\usepackage{algorithm}
\usepackage{graphics}
\usepackage{subfiles}
\usepackage{fancyhdr}
\usepackage[page]{appendix}

\ifdraft
\title{Sorry Paul}
\else
\title{An Introduction to Finite Element Methods for Solving Partial
Differential Equations on Arbatrairy Boundaries}
\fi
\newcommand\shortitle{FINITE ELEMENT METHODS}
\author{Arden Rasmussen}
\date{\today}

\fancyhf{}
\fancyhead[LE,RO]{\footnotesize \scshape \thepage}
\fancyhead[CO]{\footnotesize \scshape \shortitle}
\fancyhead[CE]{\footnotesize \scshape ARDEN RASMUSSEN}
\renewcommand{\headrulewidth}{0pt}
\setlength{\headheight}{10pt}
\pagestyle{fancy}

\numberwithin{equation}{section}
\definecolor{red}{RGB}{244,67,54}
\definecolor{pink}{RGB}{233,30,99}
\definecolor{purple}{RGB}{156,39,176}
\definecolor{deep-purple}{RGB}{103,58,183}
\definecolor{indigo}{RGB}{117,125,232}
\definecolor{blue}{RGB}{110,198,255}
\definecolor{pale-blue}{RGB}{103,218,255}
\definecolor{cyan}{RGB}{98,239,255}
\definecolor{teal}{RGB}{82,199,184}
\definecolor{green}{RGB}{128,226,126}
\definecolor{pale-green}{RGB}{190,246,122}
\definecolor{lime}{RGB}{255,255,110}
\definecolor{yellow}{RGB}{255,255,114}
\definecolor{amber}{RGB}{255,243,80}
\definecolor{orange}{RGB}{255,201,71}
\definecolor{deep-orange}{RGB}{255,138,80}
\definecolor{brown}{RGB}{169,130,116}

\newcommand{\rtodo}[1]{\todo[inline, color=red]{#1}}
\newcommand{\ptodo}[1]{\todo[inline, color=purple]{#1}}
\newcommand{\btodo}[1]{\todo[inline, color=blue]{#1}}
\newcommand{\gtodo}[1]{\todo[inline, color=green]{#1}}
\newcommand{\otodo}[1]{\todo[inline, color=orange]{#1}}

\newenvironment{Figure}
{\par\medskip\noindent\minipage{\linewidth}}
{\endminipage\par\medskip}

\theoremstyle{definition}
\newtheorem{example}{Example}[section]
\newtheorem{definition}{Definition}[section]

\renewcommand{\thealgorithm}{\arabic{section}.\arabic{algorithm}}
\renewcommand\thefigure{\thesection.\arabic{figure}}

\newcommand{\pder}[2][]{\frac{\partial#1}{\partial#2}}
\newcommand{\spder}[2][]{\frac{\partial^2 #1}{\partial#2^2}}
\newcommand{\pluseq}{\mathrel{+}=}
\newcommand{\minuseq}{\mathrel{-}=}

\newcommand{\N}{\mathbb{N}}
\newcommand{\R}{\mathbb{R}}

\setcounter{tocdepth}{1}

\begin{document}
\maketitle

\rtodo{Error}
\ptodo{Remove}
\btodo{Appendix/Citation}
\gtodo{Addition}
\otodo{Revise}

\begin{abstract}
  Finite Element Methods is a process that is utilized to numerically
  approximate solutions to other wise impossible partial differential equations
  provided with arbatrairy boundaries. This paper serves as a brief explination
  and introduction to the methods of finite element analysis. This paper
  provides a complete explanation of the every process involved, a reader
  should be able to implement all the necessary steps required in the analysis.
\end{abstract}

\begin{multicols}{2}
  \subfile{sections/introduction}
  \subfile{sections/problem_statement}
  \subfile{sections/boundary_conditions}
  \subfile{sections/variational_formulation}
  \subfile{sections/discretization}
  \subfile{sections/localization}
  \subfile{sections/mesh_gen}
  \subfile{sections/assembly_global}
  \subfile{sections/dirichlet}
  \subfile{sections/linear_sys}
  \subfile{sections/solving_system}
  \subfile{sections/further_iteration}
  \subfile{sections/post_processing}
  \subfile{sections/conclusion}

  \nocite{*}
  \bibliographystyle{alpha}
  \bibliography{references}
\end{multicols}

\newpage

\subfile{appendix/index}

\end{document}
