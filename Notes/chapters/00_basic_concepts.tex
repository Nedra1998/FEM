\documentclass[../FEM.tex]{subfiles}

\begin{document}
\section{Basic Concepts}
\label{sec:basic-concepts}

The finite element method provides a method for generating algorithms
for approximating the solutions of differential equations.

\subsection{Weak Formulation of Boundary Value Problem}
\label{sub:weak-formulation-of-boundary-value-problem}

Considering the two-point boundary value problem

\[
\begin{aligned}
-\sder{u}{x}=f\quad(0,1)\\
u(0)=0\quad u'(1)=0
\end{aligned}
\]

Taking some function \(v\), and finding the inner product with \(f\), we
find

\[
(f,v)=\int_{0}^{1}u'(x)v'(x)dx=a(u,v)
\]

We define a vector space \(V\) as follows

\[
V \equiv \left\{ v\in L^2(0,1):\quad a(v,v)< \infty \ \text{and}\ v(0) = 0\right\}
\]

Using this we can characterize the solution to the differential equation
as any function \(u\in V\) that satisfies the boundary condition and
that satisfies

\[
a(u,v)=(f,v)\quad \forall v\in V
\]

This is called the \emph{variational} or \emph{weak} formulation of the
differential equation. It is variational because \(v\) is allowed to
vary arbitrarily. It is called weak, because there are other ways in
which to interpret the equation with less restrictive assumptions on
\(f\).

\subsection{Ritz-Galrkin Approximation}
\label{sub:ritz-galrkin-approximation}

Let \(S \subset V\) be any (finite dimensional) subspace. Then when we
replace \(V\) in the previous declaration we find

\[
u_S \in S\quad a(u_S,v)=(f,v)\quad \forall v\in S
\]

With this approximation, it can be shown that the solution \(u_S\) must
\emph{exist} and be \emph{unique}.

\subsection{Error Estimates}
\label{error-estimates}

Observing the fundamental \emph{orthogonality} between \(u\) and \(u_S\)
implies \[
a(u-u_S,w)=0\quad \forall w\in S
\]

Now we define the energy norm as

\[
\norm{v}_E=\sqrt{a(v,v)}
\] Using this norm, the Schwarz' inequality relates the energy norm and
inner-product

\[
\abs{a(v,w)}\leq \norm{v}_E\norm{w}_E\quad \forall v,w \in V
\]

\subsection{Piecewise Polynomial Spaces}
\label{sub:piecewise-polynomial-spaces}

\subsection{Relationship to Difference Methods}
\label{sub:relationship-to-difference-methods}

\subsection{Computer Implementation of Finite Element Methods}
\label{sub:computer-implementation-of-finite-element-methods}

\subsection{Local Estimates}
\label{sub:local-estimates}

\subsection{Adaptive Approximation}
\label{sub:adaptive-approximation}

\subsection{Weighted Norm Estimates}
\label{sub:weighted-norm-estimates}
\end{document}
